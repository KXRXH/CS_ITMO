\twocolumn
\par \noindent ставленную задачу. Остается заметить, что вычисление следует прекращать после того, как $x_{n+1}$ станет равным $x_{n}$ с заданным количеством десятичных знаков.

\par Метод последовательных приближений обладает одним большим достоинством: случайная ошибка в промежуточных действиях не повлияет на величину результата, а лишь увеличит время на его получение.

\par Вообще при проведении вычислений контролю должно быть уделено самое серьезное внимание. Конечно, процесс вычислений можно повторить заново, но это вдвое увеличит общее время работы. В особо ответственных случаях приходится идти и на этот шаг, но возможны и другие методы контроля, учитывающие специфику конкретного расчета. Например, если от одного аргумента x вычисляется $\sin x$ и $\cos x$, то естественно затем подсчитать $\sin^{2}x$ + $\cos^{2}x$ и сравнить с единицей.

\section*{Еще один пример}
\par \noindent Особый интерес представляет методика, позволяющая контролировать каждый этап вычислений. Продемонстрируем ее на примере решения системы линейных алгебраических уравнений.

\par Пусть нам задана система двух уравнений с двумя неизвестными:
\begin{equation} \tag{4}
 \begin{cases}
   5x + 2y= 19,\\
   3x + 7y = 23.
 \end{cases}
\end{equation}