\twocolumn
\par \noindent ставленную задачу. Остается заметить, что вычисление следует прекращать после того, как $x_{n+1}$ станет равным $x_{n}$ с заданным количеством десятичных знаков.

\par Метод последовательных приближений обладает одним большим достоинством: случайная ошибка в промежуточных действиях не повлияет на величину результата, а лишь увеличит время на его получение.


\par Вообще при проведении вычислений контролю должно быть уделено самое серьезное внимание. Конечно, процесс вычислений можно повторить заново, но это вдвое увеличит общее время работы. В особо ответственных случаях приходится идти и на этот шаг, но возможны и другие методы контроля, учитывающие специфику конкретного расчета. Например, если от одного аргумента x вычисляется $\sin x$ и $\cos x$, то естественно затем подсчитать $\sin^{2}x$ + $\cos^{2}x$ и сравнить с единицей.


\section*{Еще один пример}

\par \noindent Особый интерес представляет методика, позволяющая контролировать каждый этап вычислений. Продемонстрируем ее на примере решения системы линейных алгебраических уравнений.

\par Пусть нам задана система двух уравнений с двумя неизвестными:

    \begin{equation} \tag{4}
     \begin{cases}
       5x + 2y= 19,\\
       3x + 7y = 23.
     \end{cases}
    \end{equation}
    
\par Запишем ее коэффициенты и свободные члены в виде таблицы:

    \begin{center}
        \begin{tabular}{|c|c|c|}
            5 & 2 & 19\\
            \hline
            3 & 7 & 23\\
        \end{tabular}
    \end{center}

\par Пополним эту таблицу \textit{контрольным столбцом}, элементы которого равны сумме элементов соответствующей строки. Получим расширенную таблицу:
    \begin{equation} \tag{5}
    \begin{tabular}{|c|c|c||c|}
        5 & 2 & 19 & 26\\
        \hline
        3 & 7 & 23 & 33\\
    \end{tabular}
    \end{equation}
\pagebreak
% Right column
\par В дальнейшем будем рассматривать таблицу (5) как изображение двух систем с одинаковыми коэффициентами и разными свободными членами: системы (4) и системы
    \begin{equation} \tag{6}
     \begin{cases}
       5x + 2y= 26,\\
       3x + 7y = 33.
     \end{cases}
    \end{equation}
\par Решая системы (4) и (6) методом Гаусса, поделим все элементы первой строки на ее первый член. Получим таблицу
    \begin{equation} \tag{7}
    \begin{tabular}{|c|c|c||c|}
        1 & 0,4 & 3,8 & 5,2\\
        \hline
        3 & 7 & 23 & 33\\
    \end{tabular}
    \end{equation}
\par Контроль состоит в том, что на любом этапе сумма первых трех чисел любой строки должна равняться четвертому числу; здесь это выполняется.

\par Далее умножим все элементы первой строки на первый член второй строки. Получим

\begin{center}
    \begin{tabular}{|c|c|c||c|}
        3 & 1,2 & 11,4 & 15,6\\
        \hline
        3 & 7 & 23 & 33\\
    \end{tabular}
\end{center}
\par Вычтем почленно из 2-й строки 1-ю:
    \begin{equation} \tag{8}
    \begin{tabular}{|c|c|c||c|}
        3 & 1,2 & 11,4 & 15,6\\
        \hline
        0 & 5,8 & 11,6 & 17,4\\
    \end{tabular}
    \end{equation}
\par Из второй строки (8) получаем

\[y = \frac{11,6}{5,8} = 2,\:y = \frac{17,4}{5,8} = 3\]

\noindent Наконец, из 1-й строки таблицы (7) находим
\begin{center}
$x=3,8 - 2 * 0,4 = 3$,\\
$x=5,2 - 3 * 0,4 = 4$
\end{center}
\par Искомое решение $x=3,\:y=2$; решение системы (6): $\Tilde{x}=4,\:\Tilde{y}=3$. Имеют место равенства

\begin{equation} \tag{9}
    \Tilde{x}=x+1,\:\Tilde{y}=y+1
\end{equation}

\noindent значит, система решена верно. Отметим, что если в окончательном ответе равенства (9) не выполняются, то или при счете имели место слишком грубые округления, или была где-