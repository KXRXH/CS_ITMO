\begin{frame}
    \small
    \frametitle{\begin{flushright}\textcolor{blue}{П}реобразование из CC-2 в CC-2$^{k}$ и обратно\end{flushright}}
    \begin{center}
        \footnotesize
            \begin{tabular}{|c|c|c|}
                \hline
                    \parbox[c][1.3cm][c]{3.5cm}{\centering \bf Двоичная \\ <-> \\ Четверичная} & \parbox[c][.8cm][c]{3.5cm}{\centering \bf Двоичная \\ <-> \\ Восьмеричная} & \parbox[c][.8cm][c]{3.5cm}{\centering \bf Двоичная \\ <-> Шестнадцатеричная}\\
                \hline
                    00 <-> 0& 000 <-> 0 & 0000 <-> 0\\
                \hline
                    01 <-> 1 & 001 <-> 1 & 0001 <-> 1\\
                \hline
                    10 <-> 2 & 010 <-> 2 & 0010 <-> 2\\
                \hline
                    11 <-> 3 & 011 <-> 3 & 0011 <-> 3\\
                \hline
                    & 100 <-> 4 & ...\\
                \hline
                    & 101 <-> 5 & 1101 <-> D\\
                \hline
                    & 110 <-> 6 & 1110 <-> E\\
                \hline
                    & 111 <-> 7 & 1111 <-> F\\
                \hline
            \end{tabular}
    \end{center}
    
    \begin{flushleft}
        \textcolor{UsrGreen}{\textbf{Пример}}: $1111110001{,}1110001_{(2)}$ = $0011\,1111\,0001{,}1110\,0010_{(2)}$ = $3F1{,}E2_{(16)}$
    \end{flushleft}
    
\end{frame}